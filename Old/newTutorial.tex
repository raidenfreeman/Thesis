\documentclass[]{article}
\usepackage{lmodern}
\usepackage{amssymb,amsmath}
\usepackage{ifxetex,ifluatex}
\usepackage{fixltx2e} % provides \textsubscript
\ifnum 0\ifxetex 1\fi\ifluatex 1\fi=0 % if pdftex
  \usepackage[T1]{fontenc}
  \usepackage[utf8]{inputenc}
\else % if luatex or xelatex
  \ifxetex
    \usepackage{mathspec}
  \else
    \usepackage{fontspec}
  \fi
  \defaultfontfeatures{Ligatures=TeX,Scale=MatchLowercase}
\fi
% use upquote if available, for straight quotes in verbatim environments
\IfFileExists{upquote.sty}{\usepackage{upquote}}{}
% use microtype if available
\IfFileExists{microtype.sty}{%
\usepackage[]{microtype}
\UseMicrotypeSet[protrusion]{basicmath} % disable protrusion for tt fonts
}{}
\PassOptionsToPackage{hyphens}{url} % url is loaded by hyperref
\usepackage[unicode=true]{hyperref}
\hypersetup{
            pdfborder={0 0 0},
            breaklinks=true}
\urlstyle{same}  % don't use monospace font for urls
\usepackage{color}
\usepackage{fancyvrb}
\newcommand{\VerbBar}{|}
\newcommand{\VERB}{\Verb[commandchars=\\\{\}]}
\DefineVerbatimEnvironment{Highlighting}{Verbatim}{commandchars=\\\{\}}
% Add ',fontsize=\small' for more characters per line
\newenvironment{Shaded}{}{}
\newcommand{\KeywordTok}[1]{\textcolor[rgb]{0.00,0.44,0.13}{\textbf{#1}}}
\newcommand{\DataTypeTok}[1]{\textcolor[rgb]{0.56,0.13,0.00}{#1}}
\newcommand{\DecValTok}[1]{\textcolor[rgb]{0.25,0.63,0.44}{#1}}
\newcommand{\BaseNTok}[1]{\textcolor[rgb]{0.25,0.63,0.44}{#1}}
\newcommand{\FloatTok}[1]{\textcolor[rgb]{0.25,0.63,0.44}{#1}}
\newcommand{\ConstantTok}[1]{\textcolor[rgb]{0.53,0.00,0.00}{#1}}
\newcommand{\CharTok}[1]{\textcolor[rgb]{0.25,0.44,0.63}{#1}}
\newcommand{\SpecialCharTok}[1]{\textcolor[rgb]{0.25,0.44,0.63}{#1}}
\newcommand{\StringTok}[1]{\textcolor[rgb]{0.25,0.44,0.63}{#1}}
\newcommand{\VerbatimStringTok}[1]{\textcolor[rgb]{0.25,0.44,0.63}{#1}}
\newcommand{\SpecialStringTok}[1]{\textcolor[rgb]{0.73,0.40,0.53}{#1}}
\newcommand{\ImportTok}[1]{#1}
\newcommand{\CommentTok}[1]{\textcolor[rgb]{0.38,0.63,0.69}{\textit{#1}}}
\newcommand{\DocumentationTok}[1]{\textcolor[rgb]{0.73,0.13,0.13}{\textit{#1}}}
\newcommand{\AnnotationTok}[1]{\textcolor[rgb]{0.38,0.63,0.69}{\textbf{\textit{#1}}}}
\newcommand{\CommentVarTok}[1]{\textcolor[rgb]{0.38,0.63,0.69}{\textbf{\textit{#1}}}}
\newcommand{\OtherTok}[1]{\textcolor[rgb]{0.00,0.44,0.13}{#1}}
\newcommand{\FunctionTok}[1]{\textcolor[rgb]{0.02,0.16,0.49}{#1}}
\newcommand{\VariableTok}[1]{\textcolor[rgb]{0.10,0.09,0.49}{#1}}
\newcommand{\ControlFlowTok}[1]{\textcolor[rgb]{0.00,0.44,0.13}{\textbf{#1}}}
\newcommand{\OperatorTok}[1]{\textcolor[rgb]{0.40,0.40,0.40}{#1}}
\newcommand{\BuiltInTok}[1]{#1}
\newcommand{\ExtensionTok}[1]{#1}
\newcommand{\PreprocessorTok}[1]{\textcolor[rgb]{0.74,0.48,0.00}{#1}}
\newcommand{\AttributeTok}[1]{\textcolor[rgb]{0.49,0.56,0.16}{#1}}
\newcommand{\RegionMarkerTok}[1]{#1}
\newcommand{\InformationTok}[1]{\textcolor[rgb]{0.38,0.63,0.69}{\textbf{\textit{#1}}}}
\newcommand{\WarningTok}[1]{\textcolor[rgb]{0.38,0.63,0.69}{\textbf{\textit{#1}}}}
\newcommand{\AlertTok}[1]{\textcolor[rgb]{1.00,0.00,0.00}{\textbf{#1}}}
\newcommand{\ErrorTok}[1]{\textcolor[rgb]{1.00,0.00,0.00}{\textbf{#1}}}
\newcommand{\NormalTok}[1]{#1}
\IfFileExists{parskip.sty}{%
\usepackage{parskip}
}{% else
\setlength{\parindent}{0pt}
\setlength{\parskip}{6pt plus 2pt minus 1pt}
}
\setlength{\emergencystretch}{3em}  % prevent overfull lines
\providecommand{\tightlist}{%
  \setlength{\itemsep}{0pt}\setlength{\parskip}{0pt}}
\setcounter{secnumdepth}{0}
% Redefines (sub)paragraphs to behave more like sections
\ifx\paragraph\undefined\else
\let\oldparagraph\paragraph
\renewcommand{\paragraph}[1]{\oldparagraph{#1}\mbox{}}
\fi
\ifx\subparagraph\undefined\else
\let\oldsubparagraph\subparagraph
\renewcommand{\subparagraph}[1]{\oldsubparagraph{#1}\mbox{}}
\fi

% set default figure placement to htbp
\makeatletter
\def\fps@figure{htbp}
\makeatother


\date{}

\begin{document}

\section{Benchmarking και Χρονομέτρηση
Κώδικα}\label{benchmarking-ux3baux3b1ux3b9-ux3c7ux3c1ux3bfux3bdux3bfux3bcux3adux3c4ux3c1ux3b7ux3c3ux3b7-ux3baux3ceux3b4ux3b9ux3baux3b1}

\subsection{\texorpdfstring{\texttt{@time}}{@time}}\label{time}

Για να μετρήσουμε τις επιδόσεις του κώδικά μας, σε επίπεδο χρόνου,
χρησιμοποιούμε το macro \texttt{@time}.

\textbf{Προσοχή:} Η Julia, όπως και οι περισσότερες JIT-compiled
γλώσσες, θα μεταγλωττίσουν/παράγουν ενδιάμεσο κώδικα (bytecode)
οποιαδήποτε συνάρτηση/μέθοδο, την πρώτη φορά που καλείται. Αυτό
σημαίνει, πως την πρώτη φορά που καλούμε μια συνάρτηση που γράψαμε, θα
πρέπει να περιμένουμε να μεταγλωττιστεί. Γι' αυτό, \textbf{δεν
χρονομετρούμε την 1η κλήση}.

\emph{Προσέξτε ότι, το παραπάνω ισχύει για τα πάντα, ακόμα και
ενσωματωμένες συναρτήσεις της Julia base library ή macros. Δηλαδή, ακόμα
και το \texttt{@time} την πρώτη φορά που θα κληθεί πρέπει να
μεταγλωττιστεί.}

\begin{Shaded}
\begin{Highlighting}[]
\KeywordTok{function}\NormalTok{ f(n::}\DataTypeTok{Number}\NormalTok{)}
  \KeywordTok{for}\NormalTok{ i = }\FloatTok{0}\NormalTok{:}\FloatTok{100000}
\NormalTok{    n = n+}\FloatTok{1}
  \KeywordTok{end}
\NormalTok{  n}
\KeywordTok{end}

\NormalTok{@time f(}\FloatTok{3}\NormalTok{)}

\NormalTok{@time f(}\FloatTok{3}\NormalTok{)}
\end{Highlighting}
\end{Shaded}

\begin{Shaded}
\begin{Highlighting}[]
\FloatTok{0.003120}\NormalTok{ seconds (}\FloatTok{1.36}\NormalTok{ k allocations: }\FloatTok{63.282}\NormalTok{ KB)}
\FloatTok{0.000002}\NormalTok{ seconds (}\FloatTok{5}\NormalTok{ allocations: }\FloatTok{176}\NormalTok{ bytes)}
\end{Highlighting}
\end{Shaded}

\emph{Όπως βλέπετε, στην 1η κλήση έχουμε δραματικά περισσότερο χρόνο
εκτέλεσης}

\subsection{\texorpdfstring{\texttt{tic()} \texttt{toc()} και
\texttt{@elapsed}}{tic() toc() και @elapsed}}\label{tic-toc-ux3baux3b1ux3b9-elapsed}

Στη Julia υπάρχουν και οι συναρτήσεις \texttt{tic()} και \texttt{toc()}
(όπως και στη Matlab). Επίσης υπάρχει και το macro \texttt{@elapsed} που
μετρά το χρόνο εκτέλεσης. Η κύρια διαφορά τους με το \texttt{@time}
είναι πως το δεύτερο, μας παρέχει και πληροφορίες για το memory
allocation \emph{(επίσης το \texttt{@time} επιστρέφει και το αποτέλεσμα
που υπολόγισε)}. Αυτό είναι πολύ χρήσιμο για να καταλάβουμε εύκολα αν ο
κώδικάς μας έχει κάποιο σημείο που επιδέχεται βελτιστοποίησης. Στην
πραγματικότητα, δεν έχουν διαφορά στη χρονομέτρηση, απλά το
\texttt{@time} είναι πιο χρήσιμο.

\end{document}
