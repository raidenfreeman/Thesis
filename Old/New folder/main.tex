%%%
\documentclass[11pt,a4paper,english,greek,twoside]{thesis}
\usepackage{graphicx}
\usepackage{epstopdf}
\usepackage{indentfirst}
\usepackage{verbatim}
\usepackage{amsmath}
\usepackage{amsthm}
\usepackage{amssymb}
\usepackage{latexsym}
\bibliographystyle{hellas}
\usepackage{hyphenat}
\usepackage{makeidx}
\usepackage{algpseudocode}
\usepackage{algorithm}
\usepackage[hyphens]{url}
%\usepackage[hyphenbreaks]{breakurl}
\usepackage{enumitem}
\usepackage{xspace}
\usepackage{booktabs}
\usepackage{multirow}
\usepackage{subfig}
\usepackage{tabularx}
\usepackage{listings}
\usepackage{xcolor}
\usepackage{bbding}
\usepackage{footmisc}
\captionsetup{belowskip=10pt,aboveskip=15pt}
\addto\captionsgreek{%
  \renewcommand{\indexname}{Ευρετήριο όρων}%
}
\makeindex

% 1.5 spacing
\renewcommand{\baselinestretch}{1.2}

% latin text (and greek text)
\newcommand{\tl}[1]{\textlatin{#1}}
\newcommand{\tg}[1]{\textgreek{#1}}

% typeset short english phrases
\newcommand{\en}[1]{\foreignlanguage{english}{#1}}

% typeset source code
\newcommand{\src}[1]{{\tt\en{#1}}}

% typeset a backslash
\newcommand{\bkslash}{\en{\symbol{92}}}


% Colors
\definecolor{codegreen}{rgb}{0,0.6,0}
\definecolor{codegray}{rgb}{0.5,0.5,0.5}
\definecolor{codepurple}{rgb}{0.58,0,0.82}
\definecolor{backcolour}{rgb}{0.95,0.95,0.92}

% Listing style Python Code
\lstdefinestyle{mystyle}{
    backgroundcolor=\color{backcolour},   
    commentstyle=\color{codegreen},
    keywordstyle=\color{magenta},
    numberstyle=\tiny\color{codegray},
    stringstyle=\color{codepurple},
    basicstyle=\footnotesize,
    breakatwhitespace=false,         
    breaklines=true,                 
    captionpos=b,                    
    keepspaces=true,                 
    numbers=left,                    
    numbersep=5pt,                  
    showspaces=false,                
    showstringspaces=false,
%    showtabs=false,                  
    tabsize=2
}

% For JSON listing
\newcommand\JSONnumbervaluestyle{\color{blue}}
\newcommand\JSONstringvaluestyle{\color{red}}
\makeatletter
% switch used as state variable
\newif\ifcolonfoundonthisline
% style
\lstdefinestyle{json}
{
  backgroundcolor=\color{backcolour},
  showstringspaces    = false,
  keywords            = {false,true},
  alsoletter          = 0123456789.,
  morestring          = [s]{"}{"},
  stringstyle         = \ifcolonfoundonthisline\JSONstringvaluestyle\fi,
  MoreSelectCharTable =%
    \lst@DefSaveDef{`:}\colon@json{\processColon@json},
  basicstyle          = \ttfamily,
  keywordstyle        = \ttfamily\bfseries,
}
% flip the switch if a colon is found in Pmode
\newcommand\processColon@json{%
  \colon@json%
  \ifnum\lst@mode=\lst@Pmode%
    \global\colonfoundonthislinetrue%
  \fi
}

\lst@AddToHook{Output}{%
  \ifcolonfoundonthisline%
    \ifnum\lst@mode=\lst@Pmode%
      \def\lst@thestyle{\JSONnumbervaluestyle}%
    \fi
  \fi
  %override by keyword style if a keyword is detected!
  \lsthk@DetectKeywords% 
}

% reset the switch at the end of line
\lst@AddToHook{EOL}%
  {\global\colonfoundonthislinefalse}

\makeatother




\newtheorem{proposition}{Πρόταση}
\newtheorem{theorem}{Θεώρημα}
\newtheorem{corollary}{Συμπέρασμα}
\newtheorem{lemma}{Λήμμα}
\newtheorem{example}{Παράδειγμα}
\newtheorem{remark}{Σημείωση}
\newtheorem{notation}{Συμβολισμός}
\newtheorem{law}{Νόμος}
\renewcommand{\thedefinition}{\arabic{chapter}.\arabic{definition}}
\renewcommand{\theproposition}{\arabic{chapter}.\arabic{proposition}}
\renewcommand{\thetheorem}{\arabic{chapter}.\arabic{theorem}}
\renewcommand{\thecorollary}{\arabic{chapter}.\arabic{corollary}}
\renewcommand{\thelemma}{\arabic{chapter}.\arabic{lemma}}
\renewcommand{\theexample}{\arabic{chapter}.\arabic{example}}
\newcommand{\set}[1]{\left\{#1\right\}}
\newcommand{\To}{\Longrightarrow}

\renewcommand\lstlistingname{\tg{Κώδικας}}
\renewcommand\lstlistlistingname{\tg{Παραδείγματα Κώδικα}}
\renewcommand{\listalgorithmname}{Λίστα Αλγορίθμων}

\selectlanguage{greek}

\hyphenation{ο-ποί-α}

%%%%%%%%%%%%%%%%%%%%%%%%%%%%%%%%%%%%%%%%%%%%%%%%%%%%%
%% THESIS INFO 
%%
%
% Τίτλος Πτυχιακής Εργασίας
	\title{Τίτλος Πτυχιακής Εργασίας}
% "του" ή "της", ανάλογα με το φύλο του σπουδαστή
	\edef\toutis{του}
% Ονοματεπώνυμο σπουδαστή (ΚΕΦΑΛΑΙΑ, γενική πτώση)
	\edef\authorNameCapital{Ονοματεπώνυμο σπουδαστή}
% Ονοματεπώνυμο σπουδαστή (πεζά, ονομαστική πτώση)
	\author{Ονοματεπώνυμο σπουδαστή}
% Ονοματεπώνυμο Επιβλέποντα Καθηγητή
	\supervisor{Ονοματεπώνυμο}
% Τίτλος Επιβλέποντα Καθηγητή (πχ καθηγητής, λέκτορας κτλ)
	\edef\supervisorTitle{Τίτλος}
%% "Επιβλέπων" ή "Επιβλέπουσα", ανάλογα με το φύλο του Επιβλέποντα Καθηγητή
	\edef\supervisorMaleFemale{Επιβλέπων}
% Ονοματεπώνυμο Συνεπιβλέποντα Καθηγητή
	\covisor{Ονοματεπώνυμο}
% Τίτλος Επιβλέποντα Καθηγητή
	\edef\covisorTitle{Τίτλος}
% "Συνεπιβλέπων" ή "Συνεπιβλέπουσα", ανάλογα με το φύλο του Καθηγητή
	\edef\covisorMaleFemale{Συνεπιβλέπων}
% Ονοματεπώνυμο Συνεπιβλέποντα Καθηγητή
	\covisorS{Ονοματεπώνυμο}
% Τίτλος Συνεπιβλέποντα Καθηγητή
	\edef\covisorSTitle{Τίτλος}
% "Συνεπιβλέπων" ή "Συνεπιβλέπουσα", ανάλογα με το φύλο του Καθηγητή
	\edef\covisorSMaleFemale{Συνεπιβλέπων}
% Τόπος, μήνας και έτος
	\edef\thesisPlaceDate{Τόπος, μήνας και έτος}
% Ημερομηνία Εξέτασης
	\edef\examinationDate{Ημερομηνία Εξέτασης}
% Έτος Copyright
	\edef\copyrightYear{Έτος}
% Ονοματεπώνυμο 1ου εξεταστή
	\epitropiF{Ονοματεπώνυμο}
% Τίτλος 1ου εξεταστή
	\edef\epitropiFTitle{Τίτλος}


%%%%%%%%%%%%%%%%%%%%%%%%%%%%%%%%%%%%%%%%%%%%%%%%%%%%%


\begin{document}
\selectlanguage{greek}
\maketitle

\frontmatter	
% Ευχαριστίες
%	\include{acknowledgements}
% Περίληψη
% \include{abstract}
% Αφιέρωση
%   \thesisDedication{Αφιέρωση}
% Πίνακας Περιεχομένων
%	\tableofcontents
% Κατάλογος Σχημάτων
%	\listoffigures
%	\listofalgorithms
% Κατάλογος Πινάκων
%	\listoftables
% Κατάλογος listings
%\clearpage


\mainmatter
%%%%%%%%%%%%%%%%%%%%%%%%%%%%%%%%%%%%%%%%%%%%%%%%%%%%%
%% INCLUDE YOUR CHAPTERS/SECTIONS HERE
%%

% Εισαγωγή
%    \include{intro}
% Κεφάλαια
%    \include{chap2}
%	\include{chap3}

%	\cleardoublepage
% Βιβλιογραφία - Αναφορές
%	\include{bibliography}

% Συντομογραφίες - Αρκτικόλεξα - Ακρωνύμια
%	\include{abbreviations}

%    \cleardoublepage
	\title{Το περιβάλλον Julia: Μελέτη, αξιολόγηση και
εφαρμογές}\label{ux3c4ux3bf-ux3c0ux3b5ux3c1ux3b9ux3b2ux3acux3bbux3bbux3bfux3bd-julia-ux3bcux3b5ux3bbux3adux3c4ux3b7-ux3b1ux3beux3b9ux3bfux3bbux3ccux3b3ux3b7ux3c3ux3b7-ux3baux3b1ux3b9-ux3b5ux3c6ux3b1ux3c1ux3bcux3bfux3b3ux3adux3c2}

\subsection{Abstract}\label{abstract}

ellinika?

\subsection{Description of the Julia programming
language}\label{description-of-the-julia-programming-language}

Julia is a high level, multi-paradigm, dynamically typed, programming
language. It is aimed at fields such as numerical analysis,
computational science, while also being well suited for general purpose
programming.

Focusing on providing Ruby's dynamic types, the syntactic simplicity of
Python and C-like performance, Julia could prove to be the leading
language in high performance computing, and an indispensable tool for
research in scientific and engineering fields.

Julia is released under the MIT license, therefore is free and open
source.

\paragraph{Feature overview}\label{feature-overview}

Julia is JIT-compiled and garbage collected and uses multiple dispatch.
It is designed with high performance in mind, being comparable to much
lower level languages, such as C. In addition, parallel execution and
distributed computing are first class citizens. Other key components,
include macros and metaprogramming support, a built in package manager,
seamless interop with C and Fortran and a highly sophisticated compiler,
able to generate specialized code, depending on argument types.

\paragraph{History}\label{history}

Julia was designed by Viral B. Shah, Jeff Bezanson, Stefan Karpinski and
Alan Edelman. Released in 2015, after first being revealed on
Valentine's Day of 2012. Its user base has been growing exponentially,
while its popularity landed it at the top 50 of the TIOBE Programming
Community Index (www.tiobe.com).

\paragraph{The two languages problem}\label{the-two-languages-problem}

Julia's design, came as an answer to the two languages problem, faced by
modern data scientists; Writing a code prototype in a dynamically typed
language, to verify a working solution, but then having to rewrite a
whole new implementation in another, statically typed language in order
to achieve acceptable performance.

One can easily implement some algorithm or conceptual solution in Julia.
Its great advantage in comparison to languages like Python, is that the
very same code, can achieve the performance of highly optimized, machine
specific code (thanks to LLVM), only by introducing very minor changes,
in the form of type declaration for a method's arguments. This makes the
code extremely easy to optimize, even for users with little
understanding of low level architecture.

This is achieved, thanks to Julia's versatile and advanced compiler,
that can produce LLVM IR (intermediate representation), specialized on
the types of the parameters of each calculation. If the types are not
known in advance, the generated assembly may not make any assumptions
about the arguments' memory representation, and while perfectly working,
it is sub optimal. In case that constraints are enforced on the types,
the compiler is smart enough, to take advantage of them, and generate
assembly similar to that of the statically typed C. As a result, you
have all of the benefits of a statically typed language, both in type
safety and performance, as an opt in feature, allowing the liberty and
ease of use of a dynamic type system wherever speed is not a concern.

\subsubsection{Platforms}\label{platforms}

Julia is JIT-compiled with an LLVM backend. It can generate native code
for all of the major modern platforms:

\begin{itemize}
\tightlist
\item
  Windows
\item
  Linux
\item
  Mac OS
\item
  FreeBSD
\end{itemize}

The main architectures supported are x86-64 (AMD64) and x86, while there
is experimental support for ARM, AARCH64, and POWER (little-endian).

\subsubsection{Language Features in
depth}\label{language-features-in-depth}

\paragraph{Type System}\label{type-system}

Julia features a dynamic, nominative and parametric type system. Every
type is abstract, except concrete types, which are final. This means
that the only types that can be instantiated, are unable to have
subtypes, while their supertypes cannot be instantiated. That
distinction from other object oriented languages' type systems, implies
that inheritable types are only used to infer common behavior, not
implementation details such as data (much like interfaces in other OO
programming languages); Only the types which are at the bottom of the
type tree, can have implementations and specific memory representation.

As the type system is dynamic, usually it is not necessary to explicitly
declare types. Variables are just a way to name (or reference) data, and
do not have types, only the values that they can hold have types. As a
result of those two points, most programs do not need to declare any
type information, leading to adaptable code, that can behave robustly,
no matter the input.

\paragraph{Multiple Dispatch}\label{multiple-dispatch}

\subparagraph{Short introduction to multiple
dispatch}\label{short-introduction-to-multiple-dispatch}

Possibly the most attractive feature of Julia, is multiple dispatch, the
ability to dispatch a function, based on all of its parameters.

Multiple dispatch, is a generalization of single-dispatch polymorphism,
present in most object oriented programming languages.

Commonly, object oriented languages, have a feature called subtyping
(usually implemented through inheritance), where one parameter type, can
be substituted by its subtypes, types that are more specific and lower
in the type hierarchy tree. In every function call, the specific
function that will be executed, is selected based on the type of one of
the arguments (most commonly the first one):

\begin{Shaded}
\begin{Highlighting}[]
\NormalTok{objectOfTypeA.method()}
\end{Highlighting}
\end{Shaded}

The above, in languages like C++, C\#, Java, etc., assumes the existence
of a function named \texttt{method}, which implies an argument of some
type, named \texttt{objectOfTypeA}. The same identifier,
\texttt{method}, can also exist in other types, or even subtypes of the
aforementioned object's class. Thus, in order to differentiate them, the
type of the object ``calling'' the method, is of significance; this is
usually implemented as the first argument of a function like so:

\begin{Shaded}
\begin{Highlighting}[]
\NormalTok{method(A parameter1)}
\end{Highlighting}
\end{Shaded}

(where A is the type of the first parameter), in C like code. That is
how a function is dispatched (its code is selected for execution), in
single dispatch languages.

In contrast to that, multiple dispatch languages, do not have any
significant parameters, whose types are treated differently; They are
dispatched based on the derived type of every argument. For example:

\begin{Shaded}
\begin{Highlighting}[]
\NormalTok{method (T_1 parameter1, T_2 parameter2)}
\end{Highlighting}
\end{Shaded}

\begin{Shaded}
\begin{Highlighting}[]
\NormalTok{method (T_1 parameter1, T_3 parameter2)}
\end{Highlighting}
\end{Shaded}

Those two methods, are different, even though they have the same name,
as the type of their second parameter is T\_2 and T\_3 respectively. If
one would call those methods in the following way:

\begin{Shaded}
\begin{Highlighting}[]
\NormalTok{method (object1, object2)}
\end{Highlighting}
\end{Shaded}

The type of object 2 must be determined in order to decide which method
to route this statement to. If object 2 is of type T\_2, the first
method will be dispatched. In the special case where T\_3 is a subtype
of T\_2, if the object is of type T\_3, the second method will be
selected, as it is more specific. This is something that can only occur
for a special argument in single dispatch languages, while in multiple
dispatch, any number of arguments enjoys the same polymorphism.

\subparagraph{Method calls in Julia}\label{method-calls-in-julia}

Given the following call:

\begin{Shaded}
\begin{Highlighting}[]
\NormalTok{method(argument1, argument2)}
\end{Highlighting}
\end{Shaded}

The existence of a method named \texttt{method} is assumed, whose return
type is \texttt{T}, and takes two arguments, of types \texttt{T1} and
\texttt{T2}, named \texttt{argument1} and \texttt{argument2}
respectively.

According to the Julia spec, every function, is a member of a certain
type; This is the return type of the function, in our example
\texttt{T}. Type T, has a method table, which contains all the functions
associated with it.

When the JIT compiler encounters that statement, it will dispatch a
method through the following algorithm:

\begin{enumerate}
\def\labelenumi{\arabic{enumi}.}
\tightlist
\item
  Determine the associated type with \texttt{method}, as \texttt{T}
\item
  Lookup in the function table of \texttt{T}, for a method with that
  name (\texttt{method} in the example)
\item
  Select the one who has two parameters of types \texttt{T1} and
  \texttt{T2}
\end{enumerate}

The compiler, will generate specific code, only for those three
aforementioned types, if it doesn't already exist. In case that we call
the same method with a new argument:

\begin{Shaded}
\begin{Highlighting}[]
\NormalTok{method(argument1, argument3::T3)}
\end{Highlighting}
\end{Shaded}

of type \texttt{T3}, unrelated to \texttt{T2}, the previous method, is
not applicable, therefore a new one will be dispatched (and compiled if
needed).

In the case where the argument types are not specified -as Julia is a
dynamically typed language and does not require type declaration-, and
no assertion can be inferred about the runtime types, multiple dispatch
cannot be utilized, because only one method can be generated, that can
make no assumptions about its arguments. The generated assembly is much
less efficient, as many precautions must be taken for memory access,
since the types are uncertain, and their size is unknown, as well as
them being values or references to values.

\paragraph{Parallel and Distributed
Computation}\label{parallel-and-distributed-computation}

\paragraph{Linear Algebra}\label{linear-algebra}

\subsubsection{Optimizing Julia Code}\label{optimizing-julia-code}

\paragraph{Profiling}\label{profiling}
	\pagebreak
	\pagestyle{empty}
\end{document}